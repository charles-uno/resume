% Charles McEachern

% Fall 2015

% Adapted from work by Alex Gude. 

% Note: This document wraps at column 80. 

% #############################################################################
% ############################################################## Document Setup
% #############################################################################

% Set the size and shape of the document. 
\documentclass[11pt,letterpaper]{article}
\usepackage[letterpaper, margin=0.8in]{geometry}
\addtolength{\topmargin}{-0.25in}

% Reduce padding on the edges of nested lists. 
\setlength{\tabcolsep}{0em}

% Disable page numbering. 
\pagestyle{empty}

% Set the font used by the document. This line has to be commented out
% sometimes -- I think this particular font isn't available for ARM processors
% (such as the one in my Chromebook). 
\usepackage{tgpagella}

% Allow colored text. 
\usepackage[usenames, dvipsnames]{xcolor}

% Allow hyperlinks, and prevent LaTeX from drawing boxes around them. 
\usepackage[hidelinks]{hyperref}

% For better handling of whitespace around macros. 
\usepackage{xspace}

% Add \sout, used to add rules to header lines. 
\usepackage[normalem]{ulem}

% Allow additional formatting of lists. 
\usepackage{enumitem}

% Add \widthof for better indentation handling. 
\usepackage{calc}

% The following packages don't seem to be necessary. Uncommenting them reduces
% the number of warnings seen during compilation, but doesn't actually affect
% the PDF. "Packages only get added. Never removed." -Alex
%\usepackage[utf8]{inputenc}
%\usepackage{mdwlist}
%\usepackage[T1]{fontenc}
%\usepackage{textcomp}

% #############################################################################
% ###################################################################### Macros
% #############################################################################

% Left-align the first argument and right-align the second. 
\newcommand{\headerrow}[2]{
  \begin{tabular*}{\linewidth}{l@{ \extracolsep{\fill} }r} #1 & #2 
  \end{tabular*}}

% Format the dash in a year range nicely. 
\newcommand{\YearRange}[2]{#1--#2}

% Section header, colored, and with a line extending to the right edge. 
\newcommand{\ResumeSection}[1]{
  \section*{ {\color{MidnightBlue}#1 \sout{\hfill} } } }

% Attach a hyperlink to a URL. 
\newcommand{\URL}[1]{\href{#1}{#1}\xspace}

% Retouch the plusses to get "C++" to look nice. 
\newcommand{\CPP}{
  C\nolinebreak[4]\hspace{-.05em}\raisebox{.22ex}{\footnotesize\bf ++}\xspace}

% #############################################################################
% ############################################################## Resume Content
% #############################################################################

\begin{document}

% =============================================================================
% ====================================================================== Header
% =============================================================================

% The header would be an appropriate place to put a link to LinkedIn -- or,
% even better, GitHub. I chose to instead link to my own website (which itself
% has a prominent links to my LinkedIn account and several projects on GitHub).
% This is mostly to show off my basic knowledge of HTML/CSS and Javascript. 

\begin{center}
  { \Huge \textbf{Charles McEachern} }

  \vspace{7px}

  Minneapolis, MN
  \ \ \textbullet
  \ \ (651) 269-9245
  \ \ \textbullet
  \ \ \URL{www.charles.uno}
  \ \ \textbullet
  \ \ \sout{ \href{mailto:ch@rles.uno}{ch@rles.uno} }

  \vspace{3px}

\end{center}

% =============================================================================
% ====================================================================== Skills
% =============================================================================

\ResumeSection{Skills}

% Indentation is set by the width of the bold text "Languages". 
\begin{description}[leftmargin=!, labelindent=\parindent, 
                    labelwidth=\widthof{\bfseries Languages}]
  \parskip=0.1em

  \item[Languages]
    Bash, \CPP, Fortran, Python, some experience with HTML/CSS and JavaScript

  \item[Tools]
    BeautifulSoup, git, \LaTeX, Linux, Matplotlib, MPI, NumPy, OpenMP

\end{description}

% =============================================================================
% ================================================================== Experience
% =============================================================================

% Note that LaTeX half-cares about whitespace. Source text can be wrapped and
% indented for legibility without affecting the PDF, but a double line break
% starts a new paragraph. 

\ResumeSection{Experience}

\begin{itemize}[leftmargin=\parindent]
  \parskip=0.1em
  \itemsep=1.5em

  \item[]
    \headerrow{ \textbf{University of Minnesota Physics Department} }
              { \textbf{Minneapolis, MN} }
    \headerrow{ \emph{Research Assistant} }
              { \emph{ \YearRange{05/2009}{04/2016} } }
    \begin{itemize}
      \item Designed and implemented a simulation of electromagnetic waves in
            Earth's magnetosphere. 
      \item Performed serial and parallel benchmarks in \CPP and Fortran.
            Restructured execution to increase speed and decrease memory use. 
      \item Analyzed hundreds of gigabytes of output to identify novel 
            patterns. Created clear, handsome plots and animations using
            Matplotlib. 
      \item Presented methods and results through posters, colloquia, and
            workshops. 
    \end{itemize}

  \item[]
    \headerrow{ \textbf{Cray Inc} }{ \textbf{St Paul, MN} }
    \headerrow{ \emph{Performance Intern} }
              { \emph{ \YearRange{11/2014}{12/2015} } }
    \begin{itemize}
      \item Deployed an ensemble of nightly tests to detect bugs in Cray's 
            performance tools suite. 
      \item Created a multi-process Python harness to configure programming
            environments, dynamically generate source code, compile and run
            tests, and parse performance reports. 
      \item Isolated bugs in the performance tools, as well as in Cray, Gnu,
            and Intel compilers. 
      \item Brainstormed test features with other members of the performance
            tools team. Improved test usefulness and efficiency in response to
            feedback. 
    \end{itemize}

  \item[]
    \headerrow{ \textbf{University of Minnesota Physics Department} }
              { \textbf{Minneapolis, MN} }
    \headerrow{ \emph{Teaching Assistant} }
              { \emph{ \YearRange{01/2011}{12/2014} } }
    \begin{itemize}
      \item Communicated detailed concepts to audiences with varied technical
            backgrounds. Adapted coaching strategies to the individual needs of
            at-risk students. 
      \item Supported students outside of class through office hours, tutoring,
            and pizza review sessions. 
      \item Coordinated work flow between professors, graduate teaching
            assistants, and undergraduate tutors as Head TA. 
      \item Developed exams, laboratory exercises, and other instructional
            materials. 
    \end{itemize}

\end{itemize}

% =============================================================================
% =================================================================== Education
% =============================================================================

\ResumeSection{Education}

\begin{itemize}[leftmargin=\parindent]
  \parskip=0.1em

  \item[]
    \headerrow{ \textbf{University of Minnesota} }{ \textbf{Minneapolis, MN} }
    \headerrow{ \emph{PhD, Space Physics, Burlaga/Arctowski Medal Fellow} }
              { \emph{ \YearRange{05/2009}{04/2016} } }

    \item[]
      \headerrow{ \textbf{St Olaf College} }{ \textbf{Northfield, MN} }
      \headerrow{ \emph{BA, Math (Distinction), Physics (Distinction), 
                 Magna Cum Laude} }
                { \emph{ \YearRange{09/2005}{05/2009} } }

\end{itemize}


% =============================================================================
% =============================================================== End of Resume
% =============================================================================

\end{document}


