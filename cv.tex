
% Charles Fyfe

% Adapted from work by Alex Gude.

% ----------------------------------------------------------------------

\documentclass[12pt,letterpaper]{article}
\usepackage[letterpaper, margin=0.8in, bottom=0.5in]{geometry}
\addtolength{\topmargin}{-0.25in}
\pagestyle{empty}

% Reduce padding on the edges of nested lists.
\setlength{\tabcolsep}{0em}

% Asterisk instead of number for footnote
\usepackage{footmisc}
\renewcommand*{\thefootnote}{\fnsymbol{footnote}}
\setcounter{footnote}{1}

% mathbb in paper title
\usepackage{amsfonts}
% Allow colored text.
\usepackage[usenames, dvipsnames]{xcolor}
% Allow hyperlinks, and prevent LaTeX from drawing boxes around them.
\usepackage[hidelinks]{hyperref}
% For better handling of whitespace around macros.
\usepackage{xspace}
% Add \sout, used to add rules to header lines.
\usepackage[normalem]{ulem}
% Allow additional formatting of lists.
\usepackage{enumitem}
% For the PDF icons. SVG handling is bugged on OSX.
\usepackage{graphicx}

% ----------------------------------------------------------------------

% Left-align the first argument and right-align the second.
\newcommand{\headerpair}[2]{
    \begin{tabular*}{\linewidth}{l@{ \extracolsep{\fill} }r} {\large\emph{#1}} & {\large\emph{#2}}
    \end{tabular*}
}

% Float four tags width-wise all the way to the edges.
% \newcommand{\floatfour}[4]{ {#1}\hfill{#2}\hfill{#3}\hfill{#4} }

% Float four elements with fixed spacing between them.
\newcommand{\floatfour}[4]{ {#1}\hbump{#2}\hbump{#3}\hbump{#4} }

\newcommand{\headerrow}[3]{\headerpair{#2: #1}{#3}}

\newcommand{\ResumeSection}[1]{\section*{{\color{MidnightBlue}#1 \sout{\hfill}}}}

% Retouch the plusses to get "C++" to look nice.
\newcommand{\CPP}{C\nolinebreak[4]\hspace{-.05em}\raisebox{.22ex}{\footnotesize\bf ++}\xspace}

% Finagle with icon spacing.
\newcommand{\hbump}{\hspace{0.4in}}
\newcommand{\hbmp}{\hspace{0.02in}}
\newcommand{\iconsize}{8px}

\newcommand{\emailtag}{\includegraphics[height=\iconsize]{icons/envelope-regular.pdf} \hbmp \href{mailto:ch@rles.uno}{ch@rles.uno}}
\newcommand{\webtag}{\includegraphics[height=\iconsize]{icons/link-solid.pdf} \hbmp \href{www.charles.uno}{charles.uno}}
\newcommand{\phonetag}{\includegraphics[height=\iconsize]{icons/phone-solid.pdf} \hbmp \href{tel:+16512699245}{651-269-9245}}
\newcommand{\citytag}{\includegraphics[height=\iconsize]{icons/map-marker-alt-solid.pdf} \hbmp Minneapolis}

% ----------------------------------------------------------------------

\begin{document}

\begin{center}

    { \huge \textbf{Charles Fyfe} }

    \vspace{12pt}

    \floatfour{\emailtag}{\webtag}{\phonetag}{\citytag}

\end{center}

% ----------------------------------------------------------------------

\ResumeSection{Experience}

\begin{itemize}[leftmargin=\parindent]
    \parskip=0.1em
    \itemsep=1.2em
    \item[]
        \headerrow
            {Intern $\to$ Software Engineer}
            {Cray Inc}
            {2014--Present}
        \begin{itemize}[leftmargin=\parindent]
            \itemsep=0.3em

            \item Rolled out a Continuous Testing solution for the Shasta product line. Secured VP buy-in, introduced requirements to 100+ engineers, and followed up face-to-face.

            \item Configured and upgraded a 200-node mainframe for a team of 20 test engineers. Prepared educational materials to onboard dozens of new employees to Cray software and hardware.

%            \item Wrote a \textbf{Groovy} pipeline to build, test, and distribute RPMs in support of 100+ engineers.

%            \item Illustrated pipeline usage with an annotated end-to-end example: a \textbf{REST API} written in \textbf{Go}, wrapped in a \textbf{Docker} container, and deployed to \textbf{Kubernetes} via \textbf{Ansible}.

            \item Led development of a pipeline to automatically build, test, and deploy software. Created annotated end-to-end examples to help employees transition into the modernized workflow.

            \item Mentored two interns, both of whom were awarded extensions.

%            \item Rolled out a \textbf{Continuous Testing} solution for the Shasta product line. Secured VP buy-in, communicated requirements to dozens of product groups, and followed up on concerns.
%            \item Designed a framework to execute nightly tests on live mainframes. Ingested results into an \textbf{ELK} database. Interfaced with \textbf{Slack} and \textbf{Jira} APIs to notify product owners of failures.
%            \item Crawled the \textbf{Jenkins REST API} to monitor deployments of 6 product streams across 27 mainframes. Aggregated metrics onto a \textbf{CGI} dashboard viewed 100+ times daily.
%            \item Implemented a Python API for access and analysis of mainframe system logs. Parsed terabytes of text to diagnose hardware failures on a \$70 million customer installation.

            \item Parsed system logs to isolate failures on in-house machines and on a \$70 million customer installation. Summarized implementation and outcome for an audience of 50+ engineers.

%            \item Prepared and presented educational materials to onboard dozens of new employees.

% Administered boots and upgrades for a 200-node Cray XC mainframe. Supported bizarre hardware and software configurations to maximize test coverage for a team of 20 engineers.


% CT stack: RPM build pipeline, teaching developers how to write/package tests, detecting and executing tests, sending results to ELK, notifying product owners of failures, aggregating health on a dashboard

% Cori has 12k+ nodes
% Crawled thousands of repos via BitBucket's REST API. Validated pipeline changes against live use cases to avoid disrupting product streams
% Triggered jobs via Jenkins REST API. Monitored the queue and throttled job submission to avoid deadlocking the build server
% Auto-detected tests at build time and registered them withTestRail via REST API
% Aggregated hardware and software status onto a CGI dashboard
% Mentored two interns, both of whom were awarded extensions.
% Prepared and presented educational materials to onboard dozens of new employees.
% Implemented a Python API for control and testing of the Cray XC cooling system. Isolated bugs that, if released, would have cost millions of dollars in waste and damage.
% Exported and analyzed mainframe logs for display in an ELK dashboard. Flagged anomalous boots in real time, allowing problems to be isolated in hours rather than days.
% Deployed a parallel \textbf{Python} harness to run nightly tests against Cray's performance analysis tools. Filed detailed bugs against Cray, Gnu, and Intel compilers.
% Automated shell environment configuration, \CPP and Fortran source code generation, and performance report parsing.


        \end{itemize}

%    \item[]
%        \headerrow
%            {Volunteer Powerlifting Coach}
%            {Special Olympics Minnesota}
%            {2018--Present}
%        \begin{itemize}[leftmargin=\parindent]
%            \item Coordinated warm-ups, lifts, meals, and parent concerns during full-day Area and State meets.
%            \item Worked with 20 athletes aged 16 to 60 to improve health, strength, and confidence.
%        \end{itemize}

    \item[]
        \headerrow
            {Teaching Assistant $\to$ Research Assistant}
            {University of Minnesota}
            {2009--2016}
        \begin{itemize}[leftmargin=\parindent]
            \itemsep=0.3em

            \item Modeled ultra low frequency Alfve\'n waves in the inner magnetosphere. Considered ring current enhancement as a novel simulation driver.
%            \item Benchmarked and optimized a model of near-Earth electromagnetic waves in parallel \textbf{Fortran}. Analyzed hundreds of gigabytes of data in \textbf{Python} to identify novel patterns.
%            \item Benchmarked and optimized a wave simulation in parallel \textbf{Fortran}. Implemented a \textbf{Python} harness to configure, compile, and launch hundreds of simulations in parallel.
            \item Analyzed hundreds of gigabytes of data to isolate novel patterns. Shared methods and results via posters, papers, and workshops.
%            \item Developed a numerical model of near-Earth electromagnetic waves in parallel \textbf{Fortran}. Analyzed hundreds of gigabytes of data in \textbf{Python} to identify novel patterns.
%            \item Created clear data visualizations using \textbf{Matplotlib}. Shared methods and results via posters, papers, and workshops.
%automated job launch and data analysis in python
            \item Led laboratory exercises and tutored at-risk students individually. Coordinated between professors, graduate teaching assistants, and undergraduate tutors as Head TA.
%Communicated detailed concepts to audiences with varied technical backgrounds.
%Coached new team members to improve student outcomes.
        \end{itemize}

\end{itemize}

% ----------------------------------------------------------------------

\ResumeSection{Volunteering}

\begin{itemize}[leftmargin=\parindent]
    \parskip=0.1em
    \itemsep=1.2em

    \item[]
        \headerrow
            {Volunteer Powerlifting Coach}
            {Special Olympics Minnesota}
            {2018--Present}
        \begin{itemize}[leftmargin=\parindent]
            \itemsep=0.3em

            \item Weekly practices, interfaced with parents, meal and warmup logistics at competitions
        \end{itemize}
    \item[]
        \headerrow
            {Shop Volunteer}
            {Science Museum of Minnesota}
            {2018--Present}
        \begin{itemize}[leftmargin=\parindent]
            \itemsep=0.3em

            \item Soldering, exhibit assembly, CircuitPython microcontrollers
        \end{itemize}
\end{itemize}

% ----------------------------------------------------------------------

\ResumeSection{Education}

\begin{itemize}[leftmargin=\parindent]
    \parskip=0.1em
    \itemsep=1.2em

    \item[]
        \headerrow
            {PhD}
            {University of Minnesota}
            {2009--2016}
        \begin{itemize}[leftmargin=\parindent]
            \item Space physics, Burlaga/Arctowski Medal Fellow. Advisor: Bob Lysak

        \end{itemize}
    \item[]
        \headerrow
            {BA}
            {St Olaf College}
            {2005--2009}
        \begin{itemize}[leftmargin=\parindent]
            \item Math (Distinction), Physics (Distinction), Magna Cum Laude. Advisor: Amy Kolan
        \end{itemize}
\end{itemize}

% ----------------------------------------------------------------------

\ResumeSection{Teaching Assistantships}

\begin{itemize}[leftmargin=\parindent]
    \parskip=0.1em
    \itemsep=1.2em

    \item[]
        \headerpair{University of Minnesota}{}
        \begin{itemize}[leftmargin=\parindent]
            \item Fall 2014: Plasma Physics
            \item Spring 2014: Introductory Physics for Science and Engineering II (Head TA)
            \item Spring 2013: Introductory College Physics II
            \item Fall 2012: Quantum Mechanics
            \item Fall 2011: Introductory Physics for Science and Engineering I
            \item Spring 2011: Introductory Physics for Science and Engineering II
%            \item Fall 2014: Physics 4621 (Bob Lysak) (Plasma Physics)
%            \item Spring 2014: Physics 1302 (Evan Frodermann) (Head TA) (Introductory Physics for Science and Engineering II)
%            \item Spring 2013: Physics 1102 (Introductory College Physics II) (Lucy Fortson)
%            \item Fall 2012: Physics 4101 (Oriol Valls) (Quantum Mechanics)
%            \item Fall 2011: Physics 1301 (Introductory Physics for Science and Engineering I) (Leon Hsu)
%            \item Spring 2011: Physics 1302 (Introductory Physics for Science and Engineering II) (Paul Haines)
        \end{itemize}

    \item[]
        \headerpair{St Olaf College}{}
        \begin{itemize}[leftmargin=\parindent]
            \item Spring 2009: Principles of Physics II
            \item Spring 2009: Ballroom Dance I
            \item Interim 2009: Musical Acoustics
            \item Fall 2008: Principles of Physics I
            \item Interim 2008: Structures of Higher Mathematics
%            \item Spring 2009: Physics 125 (Principles of Physics II) (Dave Dahl)
%            \item Spring 2009: Ballroom Dance 1 (Anne Von Bibra)
%            \item Interim 2009: Physics 252 (Musical Acoustics) (Dave Nitz)
%            \item Fall 2008: Physics 124 (Principles of Physics I) (Dave Nitz)
%            \item Interim 2008: Math 234 (Structures of Higher Mathematics) (Cliff Corzatt)
        \end{itemize}

\end{itemize}

% ----------------------------------------------------------------------

\ResumeSection{Publications}

\begin{itemize}[leftmargin=\parindent]
    \parskip=0.1em
    \itemsep=1.2em

    \item[]
        \begin{itemize}[leftmargin=\parindent]
            \item \href{http://hdl.handle.net/11299/181780}{McEachern, C.\footnotemark[1] (2016). Modeling Pc4 Pulsations in Two and a Half Dimensions with Comparisons to Van Allen Probes Observations. University of Minnesota Digital Conservancy}.
            \item \href{http://e.ijpam.eu/contents/articles/201300604003.pdf}{McEachern, C.\footnotemark[1] (2013). Computational Solutions to the Ripple Problem in Large Magic Decks. International Electronic Journal of Pure and Applied Mathematics, 6(4), 229-234.}
            \item \href{https://doi.org/10.1016/j.dam.2011.04.022}{Garrett, K. C., McEachern, C.\footnotemark[1], Frederick, T., and Hall-Holt, O. (2011). Fast computation of Andrews’ smallest part statistic and conjectured congruences. Discrete Applied Mathematics, 159(13), 1377-1380.}
            \item \href{http://e.ijpam.eu/contents/articles/201100303004.pdf}{McEachern, C.\footnotemark[1] (2011). Applying Linear Lebesgue Density to $\mathbb{R}^{n+1}$ with Concentric $n$-Spheres. International Electronic Journal of Pure and Applied Mathematics, 3(3), 217-222.}
            \item \href{https://www.sciencedirect.com/science/article/abs/pii/S0022285208001872}{Cederberg, J., Randolph, J., McDonald, B., Paulson, B., McEachern, C.\footnotemark[1] (2008). Hyperfine spectra of KBr and KI. Journal of Molecular Spectroscopy. 250. 114-116.}
        \end{itemize}

\end{itemize}

\footnotetext{Previous name}







\end{document}
