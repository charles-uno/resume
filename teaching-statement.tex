
\documentclass[12pt,letterpaper]{article}
\usepackage[letterpaper, top=0.55in]{geometry}
\pagestyle{empty}
\usepackage[parfill]{parskip}
% Allow hyperlinks, and prevent LaTeX from drawing boxes around them.
\usepackage[hidelinks]{hyperref}
% For better handling of whitespace around macros.
\usepackage{xspace}
% For the PDF icons. SVG handling is bugged on OSX.
\usepackage{graphicx}

% ----------------------------------------------------------------------

% Float four elements with fixed spacing between them.
\newcommand{\floatfour}[4]{ {#1}\hbump{#2}\hbump{#3}\hbump{#4} }

% Finagle with icon spacing.
\newcommand{\hbump}{\hspace{0.4in}}
\newcommand{\hbmp}{\hspace{0.02in}}
\newcommand{\iconsize}{8px}

\newcommand{\emailtag}{\includegraphics[height=\iconsize]{icons/envelope-regular.pdf} \hbmp \href{mailto:ch@rles.uno}{ch@rles.uno}}
\newcommand{\webtag}{\includegraphics[height=\iconsize]{icons/link-solid.pdf} \hbmp \href{www.charles.uno}{charles.uno}}
\newcommand{\phonetag}{\includegraphics[height=\iconsize]{icons/phone-solid.pdf} \hbmp \href{tel:+16512699245}{651-269-9245}}
\newcommand{\citytag}{\includegraphics[height=\iconsize]{icons/map-marker-alt-solid.pdf} \hbmp Minneapolis}

% ----------------------------------------------------------------------

\begin{document}

\begin{center}

    { \huge \textbf{Charles Fyfe} }

    \vspace{4pt}

    \floatfour{\emailtag}{\webtag}{\phonetag}{\citytag}

    \vspace{-4pt}

    \rule{\textwidth}{0.5pt}

    \vspace{24pt}

\end{center}

% ----------------------------------------------------------------------

Teaching is a relationship, and relationships are built on honesty. From the start, it's critical to be honest with students about the expectations for the homework, projects, and other deliverables. I believe it's equally important to push students to be honest about what they need to succeed. If a concept isn't clicking, even the most anxious student should feel empowered to seek help rather than fail silently. This is true whether teaching a classroom of college students, introducing peers to new technologies, or even training athletes to perform safe strong squats.

you don't have to do this, but here are the consequences of not doing it
lecture doesn't connect with everyone. book, YouTube, or just getting their hands on it
does this make sense?
what questions do you have?




agile training
transition to devops
teams writing tests instead of throwing content over the wall




Helping coach a Special Olympics powerlifting team really hammered home my appreciation for honesty-based teaching. With so much metal moving around, a bad outcome isn't a failed test or poor grade -- it's people getting hurt.

Each athlete has different abilities in terms of strength, mobility, and communication.

can't just tell them how to do it

need to understand when something is wrong

athletes with intellectual disabilities are often adults who have been treated like children their whole lives.
``Because I said so.''


honest doesn't mean tactless
don't be vindictive


I try to keep teaching simple: make sure expectations are clear, give the students a few good examples to work from, then be available to address follow-up questions.


make sure expectations are clear

lay a solid foundation. why would you care about this?
competition
innate interest

give them a good example

be available for follow-up questions


resources to succeed
be ready for different students to have different needs, even within the same classroom

encourage students be honest about what they need







% ----------------------------------------------------------------------

\end{document}
