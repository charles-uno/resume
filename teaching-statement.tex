
\documentclass[12pt,letterpaper]{article}
\usepackage[letterpaper, top=0.55in]{geometry}
\pagestyle{empty}
\usepackage[parfill]{parskip}
% Allow hyperlinks, and prevent LaTeX from drawing boxes around them.
\usepackage[hidelinks]{hyperref}
% For better handling of whitespace around macros.
\usepackage{xspace}
% For the PDF icons. SVG handling is bugged on OSX.
\usepackage{graphicx}

% ----------------------------------------------------------------------

% Float four elements with fixed spacing between them.
\newcommand{\floatfour}[4]{ {#1}\hbump{#2}\hbump{#3}\hbump{#4} }

% Finagle with icon spacing.
\newcommand{\hbump}{\hspace{0.4in}}
\newcommand{\hbmp}{\hspace{0.02in}}
\newcommand{\iconsize}{8px}

\newcommand{\emailtag}{\includegraphics[height=\iconsize]{icons/envelope-regular.pdf} \hbmp \href{mailto:ch@rles.uno}{ch@rles.uno}}
\newcommand{\webtag}{\includegraphics[height=\iconsize]{icons/link-solid.pdf} \hbmp \href{www.charles.uno}{charles.uno}}
\newcommand{\phonetag}{\includegraphics[height=\iconsize]{icons/phone-solid.pdf} \hbmp \href{tel:+16512699245}{651-269-9245}}
\newcommand{\citytag}{\includegraphics[height=\iconsize]{icons/map-marker-alt-solid.pdf} \hbmp Minneapolis}

% ----------------------------------------------------------------------

\begin{document}

\begin{center}

    { \huge \textbf{Charles Fyfe} }

    \vspace{4pt}

    \floatfour{\emailtag}{\webtag}{\phonetag}{\citytag}

    \vspace{-4pt}

    \rule{\textwidth}{0.5pt}

    \vspace{24pt}

\end{center}

% ----------------------------------------------------------------------

Teaching is a relationship, and relationships are built on honesty. From the start, I believe it's critical to be honest with students about expectations for homework, projects, and other deliverables. It's equally important to push students to be honest about what they need to succeed. If a concept isn't clicking, even the most anxious student should feel empowered to seek help rather than fail silently. This is true whether I'm teaching a classroom of college students, introducing peers to new technologies, or training athletes to perform safe strong squats.

In my experience, honest communication with college students is particularly relevant in a laboratory setting. Students often fixate on getting the ``right answer," and interpret lost points as a result of bad data. In fact, the culprit is typically vague descriptions, missing uncertainties, or incomplete analysis. By (tactfully) making sure a student understands where they fell short, and inviting them to ask for specific help, I set them up to do better next time.

The same principles apply when I'm transitioning coworkers to new technologies and workflows: standardized builds, automated testing, and regular demos. I'm not the one signing paychecks, so I can't exactly impose requirements. But I can give them an honest account of what to expect. Some engineers are intrinsically curious, and that's enough to get them going. Others drag their feet until new rules come down from the top. The best way to get them back on track is an honest account of what they know and what they don't.

For me, what's really hammered home the importance of honesty in teaching is volunteer-coaching a Special Olympics powerlifting team. With so much metal in motion, a bad outcome isn't a failed test or poor grade; it's people getting hurt. Each athlete has different abilities in terms of strength, mobility, and communication. And appeals to authority (``because I said so'') are toxic -- athletes with intellectual disabilities are often adults who have been treated like children their whole lives. That means a lot of honest talks about why you're lifting 135 when you know you can do 150, why it's important to stop as soon as I ask you to, and how we can resolve this frustration and get back to the workout.

Students can learn from a book or a video, but they do better when learning from a professor they trust. That means two-way honesty. I'm honest with my students about what I expect, and how their work lines up with those expectations. I also push them to be honest about their own needs. Whether they're struggling with a physics concept, a personal issue, or a systemic barrier, it's always better to ask for help rather than fail silently.

% ----------------------------------------------------------------------

\end{document}
